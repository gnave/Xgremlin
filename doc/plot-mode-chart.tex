
%
% a wall chart for Xgremlin's plot mode commands
%
\documentclass[10pt]{article}

\usepackage{times}

\sloppy
\pagestyle{empty}

\voffset = -1in
\hoffset = -1in
\textwidth = 6.5in
\textheight = 10in

\def\key#1#2{\textbf{#1}+\textbf{#2}}
\def\kkey#1#2#3{\textbf{#1}+\textbf{#2}+\textbf{#3}}

\begin{document}

\noindent
{\large {\bf Xgremlin plot mode commands \hfill \today}\\[1mm]

\noindent
In the following table the letters S, C and M signify the following
key modifiers: {\bf S: Shift} key, {\bf C: control} key, {\bf M: meta}
key (usually Alt on a PC).  The meaning of the left column of the
table is is best explained by the following example: \key{S}{x} stands
for: hold down the {\bf shift} key while pressing the {\bf x}
key. The {\bf TAB} key may be used to toggle between plotting and 
editing modes (and windows). \\[2mm]

\noindent
{\bf Plotting and moving around}
\begin{tabbing}
\textbf{xxxxxxxx}\=
xxxxxxxxxxxxxxxxxxxxxxxxxxxxxxxxxxxxxxxxxxxxxxxxxxxxxxxxxxxxx\= \kill
\textbf{.} 	\> 	  move to the right, do not rescale y axis\\
\textbf{$>$}    \>	  move  to the right, rescale y axis to fill plotter window\\
\textbf{,}	\>	  move to the left, do not rescale y axis\\
\textbf{$<$}	\>	  move to the left, rescale y axis to fill plotter window\\
\textbf{z}	\>	  move zero line to the centre of the plot\\
\textbf{e}	\>	  move zero line to bottom edge of the plot\\
		\>	  \\
\textbf{x}	\>	  expand x--axis around centre of window or marker, factor 0.5\\
\key{S}{x} 	\>        expand x--axis,  factor 0.25\\
\key{C}{x}	\>        expand x--axis, factor 0.1\\
\key{M}{x}	\>        compress x--axis around centre of window or marker, factor 0.5\\
		\>	  \\	
\textbf{y}	\>	  expand y--axis around zero line or marker, factor 0.5\\
\key{S}{y}	\>        expand y--axis, factor 0.25\\
\key{C}{y}	\>  	  expand y--axis, factor 0.1\\
\key{M}{y}	\>        compress y--axis around zero line (zoom out)\\
		\>	  \\	
\textbf{\^}	\>	  full scale in y direction, data fill plotting window\\
		\>	  \\	
\textbf{p}	\>        overplot with current content of r array\\
\key{C}{r}	\>        exchange r and tr, then overplot\\
\key{S}{p}	\>	  replot; put zeroline where marker is set\\
\key{S}{u}	\>	  undo last plot mode operation\\
\end{tabbing}

\noindent
{\bf Line intensities}
\begin{tabbing}
\textbf{xxxxxxxx}\=
xxxxxxxxxxxxxxxxxxxxxxxxxxxxxxxxxxxxxxxxxxxxxxxxxxxxxxxxxxxxx\= \kill
\key{M}{a}	\>	  execute an {\em area} command\\
\end{tabbing}

\noindent	
{\bf Plot modes}
\begin{tabbing}
\textbf{xxxxxxxx}\=
xxxxxxxxxxxxxxxxxxxxxxxxxxxxxxxxxxxxxxxxxxxxxxxxxxxxxxxxxxxxx\= \kill
\textbf{n}	\>	  normal plot; no complex data. r array contains real data\\
\textbf{r}	\>	  r array contains complex data, plot real part only\\
\textbf{i}	\>	  r array contains complex data, plot imaginary part only\\
\textbf{c}	\>	  r array contains complex data, plot real and imag part\\
\end{tabbing}

\newpage

\noindent	
{\bf Line list related commands}
\begin{tabbing}
\textbf{xxxxxxxx}\=
xxxxxxxxxxxxxxxxxxxxxxxxxxxxxxxxxxxxxxxxxxxxxxxxxxxxxxxxxxxxx\= \kill
\key{S}{a}	\>   make all lines in the current line buffer {\em active}\\
\key{C}{a}	\>   add the (cursor--)marked line(s) to the line list\\
\key{C}{d}	\>   delete (drop) marked line(s) from the line list\\
\key{C}{s}	\>   use last marked line for width, damping etc. of future added lines\\
\key{C}{t}	\>   toggle a line {\em active} or {\em inactive}\\
\textbf{v}      \>   move marked line position from red to blue mouse marker\\
\key{S}{v}	\>   toggle all line markers visible or invisible on the screen\\
\textbf{g}	\>   calculate centre of gravity between two markers\\
		\>   centre of gravity is not inserted into the line list\\
\key{C}{g}	\>   same as 'g' command but the resulting c.g. is also\\
		\>   inserted into the internal line list\\
\textbf{m}	\>   print information about the cursor--marked line\\
\textbf{a,o}	\>   mark all lines in the buffer with line markers\\
\textbf{w}	\>   calculate centroid of line whose edge was cursor--marked\\
\key{C}{w}	\>   same as 'w' and insert line into internal line list\\
\kkey{M}{C}{g}  \>   do a {\em getlines inactive} (read line list file)\\
\end{tabbing}

\noindent	
{\bf Phase correction}
\begin{tabbing}
\textbf{xxxxxxxx}\=
xxxxxxxxxxxxxxxxxxxxxxxxxxxxxxxxxxxxxxxxxxxxxxxxxxxxxxxxxxxxx\= \kill
\key{C}{d}      \>  mark phase points between two markers as bad \\
\key{M}{b}      \>  save bad points from buffer to bad points file \\
\end{tabbing}

\noindent	
{\bf Miscellaneous}
\begin{tabbing}
\textbf{xxxxxxxx}\=
xxxxxxxxxxxxxxxxxxxxxxxxxxxxxxxxxxxxxxxxxxxxxxxxxxxxxxxxxxxxx\= \kill
\key{S}{c}      \>  find the center of an interferogram in r and display it\\
\key{C}{z}      \>  set to zero the r-array between two mouse markers\\
\key{S}{n}	\>  put polynomial through 1-3 (cursor-)marked points, normalize to 1\\
\key{S}{b}	\>  put polynomial through 1-3 (cursor-)marked points, subtract from r\\
\textbf{\#}     \>  print out 11 data points around each cursor marker\\
\key{M}{c}	\>  connect two or more markers by a straight line, interpolate data\\
\textbf{l}      \>  replace a set marker by a wavenumber label\\
\key{M}{l}      \>  replace a set marker by a label in secondary units ({\em nm, Hz, THz})\\
\textbf{u}      \>  cycle secondary units through {\em nm}, {\em Hz} or {\em THz}\\
\key{M}{w}	\>  set (line) {\em width} to difference of two markers\\
\textbf{\@}	\>  read data centered at a previously set mouse marker\\
\end{tabbing}
	
\end{document}









